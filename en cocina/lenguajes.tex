\documentclass[twoside,10pt]{article}
\usepackage{shlists}
\usepackage[applemac]{inputenc}
\usepackage[spanish]{babel}
\usepackage[T1]{fontenc}


\usepackage{multicol}
\usepackage{picinpar}

\usepackage{url}
\newcommand{\surl}[1]{{\small\url{#1}}}

\newcounter{vol}
\newcounter{num}
\newcounter{anyo}
\setcounter{vol}{9}
\setcounter{num}{2}
\setcounter{anyo}{2016}
\newcommand{\mes}{Mayo}
\usepackage{revisionNLcol}


\title{\ \\ Docencia 2.0\\ \large Juan Juli\'{a}n Merelo, Fernando Tricas}
\author{\LARGE Lenguajes de programación: ¿uno, ninguno o todos?}

\date{}

\AutTit{Docencia 2.0}

\begin{document}
\addtocounter{page}{2}

\maketitle
\vspace*{-5ex}

\begin{multicols}{2}

Si hay un tema universal en la informática son los lenguajes de
programación. Desde los niveles más bajos de la arquitectura a los más
altos, la forma universal en la que el ser humano se comunica con un
ordenador son los programas, escritos en alguno de los lenguajes de
programación disponibles. Los lenguajes suelen tener una serie de
características comunes: una sintaxis rígida, implementaciones (más o
menos) libres y una evolución continua.
% No se muy bien a qué se refiere lo de implementaciones libres en este
% caso

En esta última característica está uno de los problemas. El lenguaje
que enseñaste en primero puede no parecerse en nada al que usarás en
el trabajo de fin de grado. Pero si consideramos que la evolución
individual está acompañada de una evolución colectiva de los tipos de
lenguajes que se usan y que, por tanto, se necesitan no ya para
conseguir un puesto de trabajo, sino siquiera para poder trabajar en
un entorno de computación determinado, la elección de un lenguaje de
programación para una asignatura o para una carrera entera se convierte en
una tarea ardua o imposible.

Porque, seamos prácticos, elegir un solo lenguaje para regirlos a
todos es imposible en nuestro país. Ni en ninguno, para el caso.
Nuestra idiosincrasia impediría que más de dos personas se pusieran de
acuerdo en qué lenguaje usar desde primero a cuarto, y en el caso
improbable de que un {\em ukase} de la superioridad impusiera uno, esa
misma idiosincrasia haría que el profesor de prácticas de 3º usara
eventualmente el que le viniera en gana. 
%Ahí va la mía
Seguramente esto es hasta sano y saludable porque, ¿quién nos aseguraría
que en el `mundo real' la sociedad y la industria había elegido este
supuesto lenguaje ganador y tan perfecto para habernos puesto de acuerdo a
nosotros? Será interesante haber tenido exposición a diversos lenguajes y
entornos de trabajo.

Descartada esa opción.

También podríamos pensar, siguiendo las metodologías modernas, la idea de
adaptación al estudiante: cada cuál que elija su favorito y que lo utilice
para desarrollar sus proyectos. Seguramente nos encontraríamos con algunas
dificultades: lenguajes poco adecuados para las tareas que están
relacionadas con la temática de la asignatura, excesivo `monocultivo' (es
fácil ponerse de acuerdo con uno mismo y terminar haciendo todo con una
sola tecnología, perdiendo la oportnidad de explorar otras) y la no
despreciable complejidad de prestar asistencia en caso de alguna cosa no
vaya bien. 
Desde nuestra experiencia esta puede ser una buena aproximación cuando
vamos avanzando en la carrera (sugerir, por ejemplo, un lenguaje que
nosotros creamos adecuado pero facilitar que se elijan alternativas).

Vamos avanzando.

¿Y en primero? 
Lo que siempre se dice en estos casos es que se trata de aprender a
programar (bases, conceptos, organización, ...), y que el lenguaje no es lo
importante. Pero luego todo el mundo tiene argumentos para defender unos
y desdeñar otros.
%--------------------------
\noindent\rule{86mm}{1pt}
\vspace{1ex} {\small{\begin{window}[0,r,\includegraphics[width = 27mm]{JJM.jpg},] 
\noindent\emph{JJ Merelo} es catedr\'{a}tico de Universidad
en el \'area de Arquitectura y Tecnolog\'{\i}a de Computadores, y
actualmente director de la Oficina de Software Libre de la UGR.
Mantiene un blog desde el a\~no 2002, y lo ha utilizado en clase desde
el a\~no 2004; tambi\'en wikis, agregadores y repositorios de c\'odigo
como herramientas docente. \'{U}ltimamente le ha dado por el \textsl{flipped
learning}, de lo que se informar\'{a} debidamente en esta columna.
\end{window}}}

\medskip

{\small{\begin{window}[0,r,\includegraphics[width = 27 mm]{FTricas1.jpg},]
		\noindent \emph{Fernando Tricas Garc\'{\i}a} es profesor
		titular de Lenguajes y Sistemas Inform\'{a}ticos del Departamento
		de Inform\'{a}tica e Ingenier\'{\i}a de Sistemas de la Universidad de
		Zaragoza.  Empez\'{o} a estudiar la blogosfera casi cuando a\'{u}n no
		exist\'{\i}a (all\'{a} por el a\~{n}o 2002) y a tratar de integrarla en los
		cursos y tareas docentes un poco despu\'{e}s.  Ha impartido
		numerosas charlas relacionadas con el tema de la Web 2.0, 

		internet y universidad,\ldots\ 
		Es actualmente Vicerrector de Tecnolog\'{\i}as de la Informaci\'{o}n y
de la Comunicaci\'{o}n.   
		\end{window}}}
%-------------------------------------------------


\noindent  
\bigskip

\noindent\emph{Todas las columnas de la serie Docencia 2.0
pueden descargarse en formato LaTeX desde
\surl{https://github.com/ReVision-Docencia-20/Columnas}}

\noindent\rule{90mm}{1pt}

{\small \noindent\copyright 2016 JJ. Merelo, F. Tricas. Este art\'{\i}culo es de acceso libre distribuido bajo los t\'{e}rminos
de la Licencia Creative Commons de Atribuci\'{o}n, que permite copiar,
distribuir y comunicar p\'{u}blicamente la obra en cualquier medio, s\'{o}lido
o electr\'{o}nico, siempre que se acrediten a los autores y fuentes
originales}

\end{multicols}
\end{document}
