\documentclass[twoside,10pt]{article}
\usepackage{shlists}
\usepackage[utf8]{inputenc}
\usepackage[spanish]{babel}
\usepackage[T1]{fontenc}


\usepackage{multicol}
\usepackage{picinpar}

\usepackage{url}
\newcommand{\surl}[1]{{\small\url{#1}}}

\newcounter{vol}
\newcounter{num}
\newcounter{anyo}
\setcounter{vol}{10}
\setcounter{num}{1}
\setcounter{anyo}{2017}
\newcommand{\mes}{Enero}
\usepackage{revisionNLcol}


\title{\ \\ Docencia 2.0\\ \LARGE El cambio frente a lo permanente: la tensión en la enseñanza de la informática}
\author{\large Juan Julián Merelo, Fernando Tricas}

\date{}

\AutTit{Docencia 2.0}

\begin{document}
\addtocounter{page}{2}

\maketitle
\vspace*{-5ex}

\begin{multicols}{2}

	Recientemente leíamos que siguen haciendo falta\footnote{Major Banks and Parts of Federal Gov't Still Rely On COBOL, Now Scrambling To Find IT 'Cowboys' To Keep Things Afloat, \url{https://developers.slashdot.org/story/17/04/10/1441254/major-banks-and-parts-of-federal-govt-still-rely-on-cobol-now-scrambling-to-find-it-cowboys-to-keep-things-afloat}} programadores (y/o programadoras) de COBOL, Fortran, Clipper... Auténticos dinosaurios de la informática que sobreviven en sistemas que siguen funcionando tal y como fueron diseñados, con algún achaque de vez en cuando. 
Sobreviviendo, por cierto, a algunos sistemas creados en épocas intermedias que sí que han sido sustituidos de manera irremediable y que iban a ser la solución de todos nuestros problemas.
Hasta podemos comprender a la dirección estratégica de esas empresas que no se atreve a emprender un cambio que, seguramente, implicaría no sólo la actualización de la tecnología sino también cambiar algunas funcionalidades, añadir otras, y que tiene algunos riesgos evidentes.

Por otro lado, escuchamos y leemos cómo se acusa a la universidad de no impartir la última tecnología `a la moda' que salvará al mundo de los males que nos atacan.
Por no hablar de cuando afirman que el título universitario es irrelevante. Que no decimos que no sea posible conseguir un buen empleo sin título, como prueban los ejemplos. Pero de ahí a lanzar a la gente a creer que no tienen que ir a la universidad y que formarse por su cuenta es una opción para todo el mundo, hay una cierta distancia.

	En medio, toda una panorámica de titulaciones (nuevas y antiguas) que cada vez necesitan más de la informática (como mínimo, herramientas informáticas, pero en muchos casos llegar a programar o a encargar productos informáticos) y que tienen que confiar en la auto-formación o confiar que el profesorado de sus temáticas que sea más competente técnicamente dedique algunas sesiones para empezar.

Y en esa tensión estamos.

Creemos que en la universidad se tienen que dar conocimientos básicos, que permitan tener una carrera posterio larga y fructífera, y adaptarse a los sucesivos cambios tecnológicos que sin duda vendrán.
	Pero también pensamos que no sería de recibo hoy en día impartir programación utilizando COBOL o Fortan (por exagerar). No parece de recibo mantenernos permanentemente en el pasado y no estar echando un vistazo a lo que sucede en el mundo para poner buenos ejemplos, utilizar tecnologías suficientemente próximas a las que parece que se usan en la actualidad, y motivar al estudiantado en su trabajo.

	Y transmitir la idea de que (y ya lo decía Heráclito) lo único permanente es el cambio, y que más valdrá estar en disposición de seguir leyendo, estudiando, formándose, o rápidamente podemos llegar a ser prescindibles. 

%--------------------------
\noindent\rule{86mm}{1pt}
\vspace{1ex} {\small{\begin{window}[0,r,\includegraphics[width = 27mm]{JJM.jpg},] 
\noindent\emph{JJ Merelo} es catedrático de Universidad
en el área de Arquitectura y Tecnología de Computadores, y
actualmente director de la Oficina de Software Libre de la UGR.
Mantiene un blog desde el año 2002, y lo ha utilizado en clase desde
el año 2004; también wikis, agregadores y repositorios de código
como herramientas docente. Últimamente le ha dado por el \textsl{flipped
learning}, de lo que se informará debidamente en esta columna.
\end{window}}}

\medskip

{\small{\begin{window}[0,r,\includegraphics[width = 27 mm]{FTricas1.jpg},]
		\noindent \emph{Fernando Tricas García} es profesor
		titular de Lenguajes y Sistemas Informáticos del Departamento
		de Informática e Ingeniería de Sistemas de la Universidad de
		Zaragoza.  Empezó a estudiar la blogosfera casi cuando aún no
		existía (allá por el año 2002) y a tratar de integrarla en los
		cursos y tareas docentes un poco después.  Ha impartido
		numerosas charlas relacionadas con el tema de la Web 2.0, 
		internet y universidad,\ldots\ 
		Es actualmente Vicerrector de Tecnologías de la Información y
de la Comunicación.   
		\end{window}}}
%-------------------------------------------------




\noindent 
\bigskip

	Y también que si han confiado en la universidad para formarse y adquirir unos conocimientos, a lo mejor más adelante pueden volver a pensar en nosotros para reciclarse, ponerse al día y, con ese espíritu de intercambio que venimos defendiendo desde hace tiempo en esta columna, pedirnos, obligarnos a estar al día y, en definitiva, mejorar todos.


\noindent\emph{Todas las columnas de la serie Docencia 2.0
pueden descargarse en formato LaTeX desde
\surl{https://github.com/ReVision-Docencia-20/Columnas}}

\noindent\rule{90mm}{1pt}

{\small \noindent\includegraphics[height = 4ex]{../CC.jpg} 2017 JJ. Merelo, F. Tricas. Este artículo es de acceso libre distribuido bajo los términos
de la Licencia Creative Commons de Atribución, que permite copiar,
distribuir y comunicar públicamente la obra en cualquier medio, sólido
o electrónico, siempre que se acrediten a los autores y fuentes
originales}

\end{multicols}
\end{document}
