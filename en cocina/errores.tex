\documentclass[twoside,10pt]{article}
\usepackage{shlists}
\usepackage[spanish]{babel}
\usepackage[applemac]{inputenc}
\usepackage[T1]{fontenc}

\usepackage{multicol}
\usepackage{picinpar}

\usepackage{url}
\newcommand{\surl}[1]{{\small\url{#1}}}

\newcounter{vol}
\newcounter{num}
\newcounter{anyo}
\setcounter{vol}{8}
\setcounter{num}{3}
\setcounter{anyo}{2014}
\newcommand{\mes}{Septiembre}
\usepackage{revisionNLcol}


\title{\ \\ Docencia 2.0\\ \large Juan Juli\'an Merelo, Fernando 
Tricas}
\author{\LARGE El miedo a fallar}

\date{}

\AutTit{Docencia 2.0}

\begin{document}
\addtocounter{page}{2}

\maketitle
\vspace*{-2ex}

\begin{multicols}{2}
 
En general, en la ense�anza universitaria se ense�a a acertar: a que
tu c�digo compile correctamente, pase todos los tests y finalmente
haga lo que tiene que hacer. Los contenidos, ejercicios y pr�cticas se
encaminan a que no haya ning�n fallo en ese proceso.

Pero en ingenier�a las cosas no siempre van como a uno le gustar�a que
fueran. Un espacio mal colocado o un punto y coma omitido y el
ordenador expulsa una serie de parrafadas, adem�s en ingl�s,
expresando su impotencia y su falta de comprensi�n por lo que el ser
humano est� intentando hacer. Y el problema es que, cada vez m�s, el
ser humano matriculado en un grado de ingenier�a, ante esa parrafada, s�lo sabe expresar su impotencia y su
falta de comprensi�n.

`Use the force, Luke' dec\'ia el maestro al aprendiz en aquella saga que se ha puesto de moda otra vez durante las \'ultimas semanas. Y a pesar de que tambi\'en aqu\'el otro dec\'ia aquello de `The force is strong with this one!', lo cierto es que el protagonista no dejaba de utilizar por ello las herramientas que ten\'ia disponibles (la espada l\'aser, las naves, los robots...).

As\'i que ser\'a nuestra tarea intentar convencer a nuestros estudiantes del valor de los mensajes de error: fundamentalmente en nuestro compilador favorito (o para el caso, de alguien) cuando hemos cometido alg\'un errorcillo nos lanza una parrafada que var\'ia mucho en calidad y contenido, pero que en muchos casos nos ayudar\'a a hacernos una idea de lo que est\'a pasando y en otros nos dar\'a por lo menos alguna pista de por d\'onde tenemos que empezar a mirar.

Las virtudes teologales de un desarrollador cuando aparecen los errores:

\begin{itemize}
\item Fe: el compilador sabe. Si se queja es que algo hemos hecho mal.
\item Esperanza: todo tiene remedio. Revisemos la l\'inea donde nos se\~nalan (desde el principio de la lista de errores) y sus alrededores. A veces ayuda dejarlo un rato y volver m\'as tarde.
\item Caridad: siempre encontraremos quien nos ayude. O a qui\'en ayudar. El c\'odigo fuente tiene la rara {\em virtud} de confundirnos cuando lo miramos fijamente. Esto no afecta a personas que no han estado mir\'andolo antes. 
\end{itemize}

Las podemos complementar con las virtudes cardinales.

\begin{itemize}
\item Prudencia. Una copia de seguridad, o un `commit' a tiempo y una nueva rama, para probar cambios sin destrozar ninguna funcionalidad. A ver si arreglando algo se nos va a estropear lo dem\'as.
\item Justicia. El compilador no tiene la culpa. El computador tampoco, ni el que nos encarg\'o la pr\'actica. 
\item Fortaleza. El que resiste gana. Y las herramientas son muy robustas, pero ... �Dejaremos que nos venzan unas pocas l\'ineas de c\'odigo?
\item Templanza. A\~nadir otra opci\'on a nuestro programa no lo hace mejor. Incluir las \'ultimas caracter\'isticas de la \'ultima versi\'on del lenguaje, tampoco. El objetivo es resolver la funcionalidad que sea necesaria y, de paso, aprender cosas interesantes. A veces tenemos la tentaci\'on de modificar c\'odigo, a\~nadir l\'ineas y duplicar funciones para resolver errores.
\end{itemize}



Y claro, `And may the Force be with you!'
\bigskip

\noindent\emph{Todas las columnas de la serie Docencia 2.0
pueden descargarse en formato LaTeX desde
\surl{https://github.com/ReVision-Docencia-20/Columnas}}

\noindent\rule{90mm}{1pt}

{\small \noindent\copyright 2015 JJ. Merelo, F. Tricas. Este art\'{\i}culo es de acceso libre distribuido bajo los t\'erminos
de la Licencia Creative Commons de Atribuci\'on, que permite copiar,
distribuir y comunicar p\'ublicamente la obra en cualquier medio, s\'olido
o electr\'onico, siempre que se acrediten a los autores y fuentes
originales}

\end{multicols}
\end{document}
	








\end{multicols}
\end{document}
 
