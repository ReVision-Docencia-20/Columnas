\documentclass[twoside,10pt]{article}
\usepackage{shlists}
\usepackage[spanish]{babel}
\usepackage[applemac]{inputenc}
\usepackage[T1]{fontenc}

\usepackage{multicol}
\usepackage{picinpar}

\usepackage{url}
\newcommand{\surl}[1]{{\small\url{#1}}}

\newcounter{vol}
\newcounter{num}
\newcounter{anyo}
\setcounter{vol}{8}
\setcounter{num}{3}
\setcounter{anyo}{2014}
\newcommand{\mes}{Septiembre}
\usepackage{revisionNLcol}


\title{\ \\ Docencia 2.0\\ \large Juan Juli\'an Merelo, Fernando 
Tricas}
\author{\LARGE El miedo a fallar}

\date{}

\AutTit{Docencia 2.0}

\begin{document}
\addtocounter{page}{2}

\maketitle
\vspace*{-2ex}

\begin{multicols}{2}
 
En general, en la ense�anza universitaria se ense�a a acertar: a que
tu c�digo compile correctamente, pase todos los tests y finalmente
haga lo que tiene que hacer. Los contenidos, ejercicios y pr�cticas se
encaminan a que no haya ning�n fallo en ese proceso.

Pero en ingenier�a las cosas no siempre van como a uno le gustar�a que
fueran. Un espacio mal colocado o un punto y coma omitido y el
ordenador expulsa una serie de parrafadas, adem�s en ingl�s,
expresando su impotencia y su falta de comprensi�n por lo que el ser
humano est� intentando hacer. Y el problema es que, cada vez m�s, el
ser humano matriculado en un grado de ingenier�a, ante esa parrafada, s�lo sabe expresar su impotencia y su
falta de comprensi�n.

\bigskip

\noindent\emph{Todas las columnas de la serie Docencia 2.0
pueden descargarse en formato LaTeX desde
\surl{https://github.com/ReVision-Docencia-20/Columnas}}

\noindent\rule{90mm}{1pt}

{\small \noindent\copyright 2015 JJ. Merelo, F. Tricas. Este art\'{\i}culo es de acceso libre distribuido bajo los t\'erminos
de la Licencia Creative Commons de Atribuci\'on, que permite copiar,
distribuir y comunicar p\'ublicamente la obra en cualquier medio, s\'olido
o electr\'onico, siempre que se acrediten a los autores y fuentes
originales}

\end{multicols}
\end{document}
	








\end{multicols}
\end{document}
 
