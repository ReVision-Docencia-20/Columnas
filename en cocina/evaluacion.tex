\documentclass[twoside,10pt]{article}
\usepackage{shlists}
\usepackage[utf8]{inputenc}
\usepackage[spanish]{babel}
\usepackage[T1]{fontenc}


\usepackage{multicol}
\usepackage{picinpar}

\usepackage{url}
\newcommand{\surl}[1]{{\small\url{#1}}}

\newcounter{vol}
\newcounter{num}
\newcounter{anyo}
\setcounter{vol}{10}
\setcounter{num}{1}
\setcounter{anyo}{2017}
\newcommand{\mes}{Enero}
\usepackage{revisionNLcol}


\title{\ \\ Docencia 2.0\\ \large Juan Julián Merelo, Fernando Tricas}
\author{\LARGE Evaluar las asignaturas de informática en el siglo XXI}

\date{}

\AutTit{Docencia 2.0}

\begin{document}
\addtocounter{page}{2}

\maketitle
\vspace*{-5ex}

\begin{multicols}{2}


%--------------------------
\noindent\rule{86mm}{1pt}
\vspace{1ex} {\small{\begin{window}[0,r,\includegraphics[width = 27mm]{JJM.jpg},] 
\noindent\emph{JJ Merelo} es catedrático de Universidad
en el área de Arquitectura y Tecnología de Computadores.
Mantiene un blog desde el año 2002, y lo ha utilizado en clase desde
el año 2004; también wikis, agregadores y repositorios de código
como herramientas docente. 
\end{window}}}

\medskip

{\small{\begin{window}[0,r,\includegraphics[width = 27 mm]{FTricas1.jpg},]
		\noindent \emph{Fernando Tricas García} es profesor
		titular de Lenguajes y Sistemas Informáticos del Departamento
		de Informática e Ingeniería de Sistemas de la Universidad de
		Zaragoza.  Empezó a estudiar la blogosfera casi cuando aún no
		existía (allá por el año 2002) y a tratar de integrarla en los
		cursos y tareas docentes un poco después.  Ha impartido
		numerosas charlas relacionadas con el tema de la Web 2.0, 
		internet y universidad,\ldots\ 
		Es actualmente Vicerrector de Tecnologías de la Información y
de la Comunicación.   
		\end{window}}}
%-------------------------------------------------

Evaluar al estudiante es una de las tareas más ingratas del docente,
en Informática o en cualquier otra carrera. No sólo es tedioso y
repetitivo, sino que es el momento en el que todas tus horas de
preparación e impartición de la asignatura se acaban reflejando en un
solo bit: el estudiante aprueba o suspende, o quizás ni siquiera se
presenta.

Al ser ingrata, es una tarea que se suele retrasar lo más
posible. También es un asunto puramente unidireccional: el profesorado
evalúa al estudiantado.

Sin embargo, la ingratitud de la tarea procede en parte del hecho
incontrovertible que la evaluación es, intrínsecamente,
bidireccional. Se evalúa un examen a la vez que el conjunto de los
exámenes y de las personas que no se han presentado evalúan al
profesor. Sin embargo, pocos compañeros dirán ``He suspendido este
examen'' al aprobar menos del 50\% de los inscritos en la
asignatura. En general, sucederá, {\em mutatis mutandis} la frase ``El
profesor me ha suspendido'', que se dirá que ``los estudiantes no
estaban bien preparados y han suspendido''. Y esto se atribuirá a
diferentes causas, la mayoría de las cuales serán, eventualmente,
responsabilidad del estudiante.

Y quizás no quede otro remedio que hacerlo así, si consideramos la
evaluación como un acto aislado del resto de los procesos de enseñanza
y aprendizaje. Pero, desgraciadamente, no es así. La evaluación, y
mucho más en el caso de la evaluación continua, es parte de un proceso
continuo de aprendizaje. Es una medición del nivel alcanzado y, como
cualquier acto de aprendizaje, una oportunidad para seguir aprendiendo
o simplemente asimilar algo aprendido. 

\noindent 
\bigskip

La evaluación es, en sí, un acto de aprendizaje, porque en un entorno
con ciertas restricciones (temporales, de acceso a material) el
estudiante tiene que poner en práctica lo que ha aprendido, lo que le
va a permitir, en muchos casos, tener epifanías del tipo ``Ahá, así
que era esto''.

Y ya que estamos hablando de la enseñanza de la informática, habría
que hacer bastante énfasis en la palabra {\em práctica}. Es evidente
que un ingeniero de caminos no va a poder construir los estribos de un
puente ferroviario durante un examen. Pero en informática tenemos a
nuestra disposición todas las herramientas para evaluar en la práctica
si se ha aprendido o no un contenido que, esencialmente, también es
práctico. Si un examen {\em teórico} de una asignatura práctica puede
ser discutible, más discutible aún es un examen {\em escrito} de las
{\em prácticas} de esa asignatura.

Y de hecho, siendo la evaluación un acto de aprendizaje, cuanto más se
acerque al mismo, más precisa va a ser y más va a redundar en la
adquisición de conocimientos, de forma autónoma, por el
estudiante. Cierto es que las clases teóricas tienen una cantidad de
personas que impiden en la práctica la enseñanza personalizada, pero
también es cierto que las clases prácticas tienen un tamaño manejable
y que todas las plataformas de enseñanza hoy en día nos proveen de
todo tipo de herramientas para hacer un seguimiento personalizado, si
no personal, de los alumnos.

Esta es la última columna del último ReVisión; pero no debería ser el
final de la innovación en la enseñanza de la informática igual que la
evaluación no debe ser el final del proceso de aprendizaje. Se debe
aprender siempre y, sobre todo, se debe de enseñar aprendiendo y de
aprender enseñando.

\noindent\emph{Todas las columnas de la serie Docencia 2.0
pueden descargarse en formato LaTeX desde
\surl{https://github.com/ReVision-Docencia-20/Columnas}}

\noindent\rule{90mm}{1pt}

{\small \noindent\includegraphics[height = 4ex]{../CC.jpg} 2018 JJ. Merelo, F. Tricas. Este artículo es de acceso libre distribuido bajo los términos
de la Licencia Creative Commons de Atribución, que permite copiar,
distribuir y comunicar públicamente la obra en cualquier medio, sólido
o electrónico, siempre que se acrediten a los autores y fuentes
originales}

\end{multicols}
\end{document}
