\documentclass[twoside,10pt]{article}
\usepackage{shlists}
\usepackage[utf8]{inputenc}
\usepackage[spanish]{babel}
\usepackage[T1]{fontenc}


\usepackage{multicol}
\usepackage{picinpar}

\usepackage{url}
\newcommand{\surl}[1]{{\small\url{#1}}}

\newcounter{vol}
\newcounter{num}
\newcounter{anyo}
\setcounter{vol}{10}
\setcounter{num}{1}
\setcounter{anyo}{2017}
\newcommand{\mes}{Enero}
\usepackage{revisionNLcol}


\title{\ \\ Docencia 2.0\\ \large Juan Julián Merelo, Fernando Tricas}
\author{\LARGE Evaluar las asignaturas de informática en el siglo XXI}

\date{}

\AutTit{Docencia 2.0}

\begin{document}
\addtocounter{page}{2}

\maketitle
\vspace*{-5ex}

\begin{multicols}{2}



Evaluar al estudiante es una de las tareas más ingratas del docente,
en Informática o en cualquier otra carrera. No sólo es tedioso y
repetitivo, sino que es el momento en el que todas las horas de
preparación, impartición y estudio de la asignatura se acaban reflejando en un
solo bit: el estudiante aprueba o suspende, o quizás ni siquiera se
presenta.
Es  una actividad con una carga de responsabilidad muy importante, por las
consecuencias negativas que puede tener en las personas evaluadas. Pero
también porque, ya que la hacemos, deberíamos tratar de que tenga la máxima
utilidad posible. Para todas las partes implicadas. 

Al ser ingrata, es una tarea que se suele retrasar lo más
posible. También es un asunto puramente unidireccional: el profesorado
evalúa al estudiantado.

Sin embargo, la ingratitud de la tarea procede en parte del hecho
incontrovertible que la evaluación es, intrínsecamente,
bidireccional. Se evalúa un examen a la vez que el conjunto de los
exámenes. En cierto modo, tanto los resultados como las personas que no se
han presentado evalúan al profesor. Sin embargo, pocos compañeros dirán
``He suspendido este
examen'' al aprobar menos del 50\% de los inscritos en la
asignatura. En general, sucederá, {\em mutatis mutandis} la frase ``El
profesor me ha suspendido'', que se dirá que ``los estudiantes no
estaban bien preparados y han suspendido'', en vez de ``He suspendido
al no lograr que apruebe más del 50\% de los estudiantes''. Y esto se atribuirá a
diferentes causas, la mayoría de las cuales serán, eventualmente,
responsabilidad del estudiante. Pero esa asignación de
responsabilidades es, también, una tarea ingrata. 

Esta evaluación unidireccional (profesor evalúa estudiante), sin tener
en cuenta la otra parte (el estudiante, a través de la evaluación,
está evaluando cómo el profesor ha impartido los diferentes
conocimientos) puede ser la única posible si consideramos la
evaluación como un acto aislado del resto de los procesos de enseñanza
y aprendizaje. Pero no es así. 
La evaluación, y
mucho más en el caso de la evaluación continua, es parte de un proceso
continuo de aprendizaje.
 Es una medición del nivel alcanzado y, como
cualquier acto de aprendizaje, una oportunidad para seguir aprendiendo
o simplemente asimilar algo aprendido. 
También una medida, si somos
sinceros con nosotros mismos, de la eficacia de la transmisión de los
conceptos e ideas, cuando descubrimos por las respuestas a las 
preguntas que hemos realizado que algo no ha quedado claro del todo, o
falta, de manera más o menos generalizada, alguna idea o no se ha
alcanzado una buena comprensión. O que determinado tema no ha sido de interés
para el estudiantado. 

Pero es que además la evaluación es, en sí, un acto de aprendizaje,
porque en un entorno 
con ciertas restricciones (temporales, de acceso a material --aquí entra la faceta ingenieril de la materia: no sólo se trata de obtener la solución, sino que hay que obtenerla dentro de un contexto y con unas condiciones de contorno--) el
estudiante tiene que poner en práctica lo que ha aprendido, lo que le
va a permitir, en muchos casos, tener epifanías del tipo ``Ajá, así
que era esto''.

También hay aprendizaje si la información proporcionada al estudiante sobre tu
trabajo le sirve para corregir, mejorar y terminar de comprender los conceptos
en los que falló.
En este sentido, esa realimentación debería ser, idealmente, tan próxima a la
prueba como sea posible. El estudiante, por su 

%--------------------------
\noindent\rule{86mm}{1pt}
\vspace{1ex} {\small{\begin{window}[0,r,\includegraphics[width = 27mm]{JJM.jpg},] 
\noindent\emph{JJ Merelo} es catedrático de Universidad
en el área de Arquitectura y Tecnología de Computadores.
Mantiene un blog desde el año 2002, y lo ha utilizado en clase desde
el año 2004; también wikis, agregadores y repositorios de código
como herramientas docente. 
\end{window}}}

\medskip

{\small{\begin{window}[0,r,\includegraphics[width = 27 mm]{FTricas1.jpg},]
		\noindent \emph{Fernando Tricas García} es profesor
		titular de Lenguajes y Sistemas Informáticos del Departamento
		de Informática e Ingeniería de Sistemas de la Universidad de
		Zaragoza.  Empezó a estudiar la blogosfera casi cuando aún no
		existía (allá por el año 2002) y a tratar de integrarla en los
		cursos y tareas docentes un poco después.  Ha impartido
		numerosas charlas relacionadas con el tema de la Web 2.0, 
		internet y universidad,\ldots\ 
		Es actualmente Vicerrector de Tecnologías de la Información y
de la Comunicación.   
		\end{window}}}
%-------------------------------------------------

\noindent parte, debería tratar de obtener
tanta información como sea 
posible, bien de la corrección en sí misma, bien de
su análisis posterior.

Y ya que estamos hablando de la enseñanza de la informática, habría
que hacer bastante énfasis en la palabra {\em práctica}. Es evidente
que un ingeniero de caminos no va a poder construir los estribos de un
puente ferroviario durante un examen. Tampoco en nuestros temas será posible construir un sistema grande y complejo, pero en informática tenemos a
nuestra disposición todas las herramientas para evaluar en la práctica
si se ha aprendido o no un contenido que, esencialmente, también es
práctico. 
Si un examen {\em teórico} de una asignatura práctica puede ser discutible, más
discutible aún es un examen {\em escrito} de las {\em prácticas} de esa
asignatura, algo desgraciadamente bastante común todavía.
Sin olvidar que el acto de la evaluación puede ser tan formativo como el propio trabajo, su
presentación, análisis y la interacción en torno a las cuestiones más
interesantes de la prueba, que pueden ayudar a determinar si realmente se han
adquirido los conocimientos, pero también pueden servir para afianzarlos,
reforzarlos y complementarlos.

Y de hecho, siendo la evaluación un acto de aprendizaje, cuanto más se
acerque a este formato, más precisa va a ser y más va a redundar en la
adquisición de conocimientos, de forma autónoma, por el
estudiante. Cierto es que las clases teóricas tienen una cantidad de
personas que impiden en la práctica la enseñanza personalizada, pero
también es cierto que las clases prácticas tienen un tamaño manejable
y que todas las plataformas de enseñanza hoy en día nos proveen de
todo tipo de herramientas para hacer un seguimiento personalizado, si
no personal, de los alumnos.

Y esta es la última columna del último ReVisión; pero no debería ser el
final de la innovación en la enseñanza de la informática igual que la
evaluación no debe ser el final del proceso de aprendizaje. Se debe
aprender siempre y, sobre todo, se debe de enseñar aprendiendo y de
aprender enseñando.
Esperamos haber aprobado la evaluación y, quién sabe, si poder seguir
escribiendo sobre estos temas en otro contexto. Han sido 22 columnas
(contando esta) en las que hemos tratado de interactuar entre los
autores (con éxito, al menos en la producción de las propias columnas)
y con otras personas (con poco éxito, a la luz de la falta de
interacción y propuestas recibidas).

¡Hasta pronto!

\noindent\emph{Todas las columnas de la serie Docencia 2.0
pueden descargarse en formato LaTeX desde
\surl{https://github.com/ReVision-Docencia-20/Columnas}}

\noindent\rule{90mm}{1pt}

{\small \noindent\includegraphics[height = 4ex]{../CC.png} 2018 JJ. Merelo, F. Tricas. Este artículo es de acceso libre distribuido bajo los términos
de la Licencia Creative Commons de Atribución, que permite copiar,
distribuir y comunicar públicamente la obra en cualquier medio, sólido
o electrónico, siempre que se acrediten a los autores y fuentes
originales}

\end{multicols}
\end{document}
