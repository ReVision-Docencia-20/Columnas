
Recientemente, con la transformación de los títulos universitarios con la convergencia al espacio europeo de educación superior se están escuchando críticas a la necesidad de realizar trabajos de fin de Grado y Máster.
Queremos dedicar estas líneas a las ventajas que vemos a este tipo de trabajo que culmiman unos estudios y que deberían verse, por muchos motivos, como un paso importante y tratar de desechar la idea de que son una molestia innecesaria.
Todo ello desde la experiencia de llevar unos cuantos años impartiendo docencia en titulaciones de ingeniería, donde el proyecto de fin de carrera y luego los trabajos de fin de Grado y Máster son una costumbre establecida. También con la experiencia de haber dirigido unos cuantos de estos trabajos y formado parte de los correspondientes tribunales evaluadores.

En el principio, eran las {\em tesinas}. Un trabajo, extenso, posiblemente más académico que otra cosa, que se presentaba con chaqueta y corbata ante un serio tribunal de la cátedra más próxima a la carrera.

En las ingenierías, las tesinas devinieron proyectos. %¿Esto es cierto? ¿Ese es el origen?
 Y los proyectos, eventualmente, se convirtieron en trabajos, perdiendo ese aura vital de algo que se emprende, que avanza. Un trabajo es algo que hay que entregar, pero que puede tener o no que ver con tu vida {\em real}.

Aún así, con su extensión a muchos grados, un trabajo fin de grado o TFG es algo que está más relacionado con la madurez del alumno que con sus conocimientos técnicos; es transversal en el sentido que tiene que servir no tanto para adquirir conocimientos técnicos, sino a ponerlos en práctica en una situación tan del mundo real como el tutor y el alumno puedan permitirse. 

\begin{itemize}
\item `Party hard, work hard'. En la parte de la fiesta seguro que nuestros allegados ya tienen claro que nos defendemos con mayor o menor acierto. El trabajo de fin de estudios debe (o al menos puede) ser un trabajo que reúna una parte de los conocimientos alcanzados en los estudios. También debe (o puede) permitirnos explorar algunos temas laterales que no se hayan tratado de menra completa en lo estudios regulares; o temas novedosos con cierto `riesgo' que no sabemos si llegarán a alcanzar la consolidación dentro de la disciplina.
\item `If you can dream it, you can do it'. Salvo que tengamos un marcado espíritu emprendedor el trabajo de fin de estudios puede ser una de las últimas veces en las que podamos tomar nuestras propias decisiones sobre lo que queremos trabajar. Luego iremos a trabajar en el `mundo real'\textregistered y allí pasaremos una temporada haciendo lo que seguramente decidirán otros.
\item `Yes we can'. Después de unos cuantos años estudiando podremos demostrar (y demostrarnos) que somos capaces de abordar un proyecto más o menos grande (no tan grande, en realidad) pasando por diversos procesos que hemos aprendido en nuestros estudios: investigar, analizar, examinar alternativas, tomar decisiones, asumir sus consecuencias, ...
\item `I did it'. También somos defensores de la presentación del trabajo en sesión pública. Seguramente hemos hecho presentaciones en algunas asignaturas, incluso de nuestro propio trabajo. Pero ahora vamos a presentar ante el público y ante el tribunal (-qué mala cara ponía ese señor que estaba segundo por la izquierda cuando dijiste eso-, podrá decir nuestra Tita Eduvigis que se puso las mejores galas para este gran momento, -yo creo que no han comprendido lo que quería decir- podrá explicar nuestro padre ante la insistencia del tribunal por aclarar determinada cuestión) nuestro trabajo de varios meses: esas decisiones, dudas, tiempo invertido. Con suerte, en presencia de familiares y amistades que sentirán con nosotros la presión del momento. Y la satisfacción del resultado.
\item `We did it'. Tampoco es despreciable el valor publicitario del proceso: un tribunal impresiona. Más aún a los foráneos que no saben que aquel día teníamos unas cuantas horas de clase, la entrega de alguna revisión y una evaluación de alguno de nuestros propios trabajos. Así que allí estamos mostrando a la sociedad que nos tomamos en serio el trabajo de los estudiantes, que hacen cosas que nos pueden resultar interesantes y útiles (e incluso, en algunos casos, de las que parecen saber más que nosotros).
\item `So proud of us'. En la mayoría de los trabajos de fin de estudios el resultado es positivo. Llega el momento de felicitarnos, que nos feliciten y sentirnos orgullosos de pertenecer a la lista de titulados de nuestro centro.
\end{itemize}
