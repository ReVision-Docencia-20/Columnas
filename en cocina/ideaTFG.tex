
En el principio, eran las {\em tesinas}. Un trabajo, extenso, posiblemente más académico que otra cosa, que se presentaba con chaqueta y corbata ante un serio tribunal de la cátedra más próxima a la carrera.

En las ingenierías, las tesinas devinieron proyectos. Y los proyectos, eventualmente, se convirtieron en trabajos, perdiendo ese aura vital de algo que se emprende, que avanza. Un trabajo es algo que hay que entregar, pero que puede tener o no que ver con tu vida {\em real}.

Aún así, con su extensión a muchos grados, un trabajo fin de grado o TFG es algo que está más relacionado con la madurez del alumno que con sus conocimientos técnicos; es transversal en el sentido que tiene que servir no tanto para adquirir conocimientos técnicos, sino a ponerlos en práctica en una situación tan del mundo real como el tutor y el alumno puedan permitirse. 
